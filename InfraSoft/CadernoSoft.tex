\documentclass[12pt]{article}
\usepackage[a4paper, portrait, margin=0.8in]{geometry}
\bibliographystyle{ieeetr}

\begin{document}
\nocite{*}

\title{Caderno de InfraSoft}
\author{Marconi Gomes}

\maketitle


\section{Introdução}
    $\rightarrow$ \textbf{O que acontece quando um programa executa?} 
    \\ Ele se transforma num processo e executa instruções (\textbf{uma por vez, compartilhando tempo}). 
    \\Basicamente na seguinte sequência (Modelo de Von Neumann): FECTH $\rightarrow$ DECODE $\rightarrow$ EXECUTE $\rightarrow$ STORE.
    
    \subsection{Sistema Operacional}
    É o conjunto de softwares responsável por: 
    \\  - Tornar fácil rodar programas (ao mesmo tempo)
    \\  - Fazer os programas compartilharem memória
    \\  - Fazer a interação com os dispositivos de I/O
    \\~\\O SO pode virtualizar recursos de hardware para diversos fins, além disso ele necessita de drivers para se comunicar com o hardware (específico para cada hardware).
    \\Ainda mais, o SO oferece interfaces de comunicação com o sistema (APIs) para executar operações de forma rápida e fácil pelos programas.
    \\~\\Os SOs de \textbf{primeira geração} eram constituídos por \textbf{válvulas e painéis de programação.} Já os de \textbf{segunda geração} começaram a fazer o uso de \textbf{transistores} e eram capazes de fazer a leitura de sistemas em lote.
    \\A \textbf{terceira geração} começou a usar os \textbf{circuitos integrados}, introduzindo o conceito também de multiprogramação, os CIs foram melhoras cada vez mais em desempenho e são usados até hoje.
    \\A \textbf{quarta geração} até o presente, os computadores avançaram o suficiente para serem usados como computadores pessoais.
\bibliography{references}
\end{document} 