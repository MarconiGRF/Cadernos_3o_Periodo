\documentclass[12pt]{article}
\usepackage[a4paper, portrait, margin=1in]{geometry}
\bibliographystyle{ieeetr}

\begin{document}
\nocite{*}

\title{Caderno de InfraCOM}
\author{Marconi Gomes}

\maketitle


\section{Introdução}
    - Noções de hosts
    \\- Aplicações
    \\- Meios de comunicação (cabeado ou não)
    
    \subsection{Componentes ou comutadores e Infraestrutura}
    $\rightarrow$ Roteadores, Switches, etc...
    \\$\rightarrow$ ISPs (Internet Service Providers) conectados
    \\$\rightarrow$ Protocolos
    \\$\rightarrow$ RFCs: Request for comments (Definição: São documentos para disponibilização de protocolos públicos, gerenciados por força tarefa de engenheiros da internet.)
    \\$\rightarrow$ IETF: Internet Engineering Task Force
    
    \subsection{Protocolos}
    $\rightarrow$ O que são: Conjunto de \textbf{regras} que definem o \textbf{formato, ordem e ações} sobre a \textbf{transmissão} das mensagens enviadas e recebidas entre entidades de redes. \quad \quad
    \\Exemplo: O protocolo para abrir um site (TCP) é dado por fazer uma requisição, o servidor responde com um arquivo e etc.

    \subsection{Internet}
    $\rightarrow$ Endpoints: Hosts (Computadores e servidores).
    \\$\rightarrow$ Meios de acesso: Tipo de transmissão, se é cabeada ou não.
    \\$\rightarrow$ Núcleo: cabos interconectados.

    \subsection{Tecnologias de conexão}
    $\rightarrow$ \textbf{DSL:} Usam o mesmo cabo para transmissão de telefonia e internet (cabo com par) que é levado até o DSLAM (DSL Access Multiplexer) este que divide os dados respectivamente pelo seu tipo. Tem respectivamente US $\leq$ 2,5Mbps e DS $\leq$ 24Mbps.
    \\$\rightarrow$ \textbf{Coaxial:} Usam um único cabo coaxial para transmissão de dados de internet e TV (cada um usando faixas de frequência reservadas para cada serviço) e nas pontas são usados multiplexadores para dividir e categorizar a banda, chegando até o \textbf{cable headend}. Geralmente usam do tipo HFC assimétrico, além de compartilhamento de estrutura podendo tornar a rede mais lenta.
    \\$\rightarrow$ \textbf{Redes Residenciais:} Normalmente usam cabos Ethernet, com geralmente um AP usando padrão IEEE 802.11*
    \\$\rightarrow$ \textbf{Redes Corporativas:} Usualmente usam a mesma infraestrutura de uma rede residencial (em questão de cabos), porém usando Switches e ISPs institucionais.
    \\$\rightarrow$ \textbf{Redes Sem Fio:} Padrão WiFi IEEE 802.11, respectivamente com suas transmissões: b/g:11/54Mbps, n:até 600Mbps e ac:até 1Gbps.

    \subsection{Hosts}
    $\rightarrow$ A função de transmissão de um host é receber mensagens da aplicação (qualquer), quebra em pequenos pacotes de L bits e os transmite a uma taxa R de transmissão.
    \\$\rightarrow$ O atraso de transmissão do pacote é dado por $\frac{L (tamanho)}{R (velocidade)}$.
    \\$\rightarrow$ Os meios físicos de transmissão são os que transferem bits. Geralmente divididos por meios \textbf{guiados} (cabos) ou \textbf{não guiados} (ondas magnéticas).
    

\bibliography{references}
\end{document}