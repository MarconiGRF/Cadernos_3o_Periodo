\documentclass[12pt]{article}
\usepackage[a4paper, portrait, margin=1in]{geometry}
\bibliographystyle{ieeetr}

\begin{document}
\nocite{*}

\title{Caderno de InfraCOM}
\author{Marconi Gomes}

\maketitle


\section{Introdução}
    - Noções de hosts
    \\- Aplicações
    \\- Meios de comunicação (cabeado ou não)
    
    \subsection{Componentes ou comutadores e Infraestrutura}
    $\rightarrow$ Roteadores, Switches, etc...
    \\$\rightarrow$ ISPs (Internet Service Providers) conectados
    \\$\rightarrow$ Protocolos
    \\$\rightarrow$ RFCs: Request for comments (Definição: São documentos para disponibilização de protocolos públicos, gerenciados por força tarefa de engenheiros da internet.)
    \\$\rightarrow$ IETF: Internet Engineering Task Force
    
    \subsection{Protocolos}
    $\rightarrow$ O que são: Conjunto de \textbf{regras} que definem o \textbf{formato, ordem e ações} sobre a \textbf{transmissão} das mensagens enviadas e recebidas entre entidades de redes. \quad \quad
    \\Exemplo: O protocolo para abrir um site (TCP) é dado por fazer uma requisição, o servidor responde com um arquivo e etc.

    \subsection{Internet}
    \textbf{Definição:} São sistemas finais conectados à outras redes através de ISPs, ou seja, a internet é uma rede de redes ISPs conectados, possibilitando assim com que dois hosts possam se comunicar, pois existe um caminho entre eles.
    \\A internet é muito complexa e sua evolução foi guiada por \textbf{políticas nacionais e enconômicas}. 
    \\~\\\textbf{Pergunta:} Como podemos refazer os passos para chegar (aproximadamente) ao estado atual da internet?
    \\\textbf{Resposta:} A dissipação de ISPs (nacionais e continentais) especializados, que se comunicam com outros ISPs de mesmo tipo utilizando pontos de troca de tráfego (Internet eXchange Point - \textbf{IXP}), até para casos que um IXP não esteja disponível possa ser utilizado outro caminho de comunicação.
    \\~\\Categorizando os ISPs, ficariam da seguinte forma:
    \\- ISP Comerciais (Tier 1): Google, Embratel, etc...
    \\- Redes de provedores de conteúdo: Google, Amazon, etc... {Essas conectam a internet aos seus datacenters}
    \\- IXP: Internet eXchange Points (Conexões geralmente intercontinentais).
    \\- ISP Regionais (Nordeste, Norte, etc)
    \\- ISP de acesso (Cidades)
    \\~\\\textbf{Conceitos:}
    \\$\rightarrow$ Endpoints: Hosts (Computadores e servidores).
    \\$\rightarrow$ Meios de acesso: Tipo de transmissão, se é cabeada ou não.
    \\$\rightarrow$ Núcleo: cabos interconectados.

    \subsection{Tecnologias de conexão}
    $\rightarrow$ \textbf{DSL:} Usam o mesmo cabo para transmissão de telefonia e internet (cabo com par) que é levado até o DSLAM (DSL Access Multiplexer) este que divide os dados respectivamente pelo seu tipo. Tem respectivamente US $\leq$ 2,5Mbps e DS $\leq$ 24Mbps.
    \\$\rightarrow$ \textbf{Coaxial:} Usam um único cabo coaxial para transmissão de dados de internet e TV (cada um usando faixas de frequência reservadas para cada serviço) e nas pontas são usados multiplexadores para dividir e categorizar a banda, chegando até o \textbf{cable headend}. Geralmente usam do tipo HFC assimétrico, além de compartilhamento de estrutura podendo tornar a rede mais lenta.
    \\$\rightarrow$ \textbf{Redes Residenciais:} Normalmente usam cabos Ethernet, com geralmente um AP usando padrão IEEE 802.11*
    \\$\rightarrow$ \textbf{Redes Corporativas:} Usualmente usam a mesma infraestrutura de uma rede residencial (em questão de cabos), porém usando Switches e ISPs institucionais.
    \\$\rightarrow$ \textbf{Redes Sem Fio:} Padrão WiFi IEEE 802.11, respectivamente com suas transmissões: b/g:11/54Mbps, n:até 600Mbps e ac:até 1Gbps.

    \subsection{Hosts}
    $\rightarrow$ A função de transmissão de um host é receber mensagens da aplicação (qualquer), quebra em pequenos pacotes de L bits e os transmite a uma taxa R de transmissão.
    \\$\rightarrow$ O atraso de transmissão do pacote é dado por $\frac{L (tamanho)}{R (velocidade)}$.
    \\$\rightarrow$ Os meios físicos de transmissão são os que transferem bits. Geralmente divididos por meios \textbf{guiados} (cabos) ou \textbf{não guiados} (ondas magnéticas).
    
    \subsection{Comutação de circuitos}
    Definição: Estabelecer um caminho exclusivo (no sentido de não poder ser usado por outros dispotivos ao mesmo tempo) para a comunicação entre dois dispositivos. O segmento de circuito (ou seja, o caminho/ligação, geralmente formado por 2 ou mais encaminhadores/roteadores) fica ocioso se não estiver sendo usado pela "chamada".
    \\~\\No mundo real, para se realizar a multiplexação de frequências usa-se o \textbf{FDM - (Frequency Division Multiplexing)} ou o \textbf{TDM - (Time Division Multiplexing)}. 
    \\$\rightarrow$ O FDM trabalha de forma a dividir o sinal para transmitir as informações em cada frequência específica para dispositivos específicos. A divisão pode ser feita de forma fixa ou sob demanda, ou seja, dividir mais ou menos o canal disponível, assim no método FDM todos podem falar ao mesmo tempo.
    \\$\rightarrow$ Já no TDM, como o nome sugere, toda a frequência do meio de comunicação é usada, entretanto cada dispositivo possui um tempo limite de tempo para transmitir os dados de forma padronizada. Dessa forma, o modo TDM não permite que dois ou mais dispositivos falem ao mesmo tempo. Apesar de parecer nos dias de hoje, não é o que acontece, pois o TDM aplicado atualmente é tão rápido que o usuário não percebe a divisão de tempo.
    \\\textbf{Observação:} O meio de comunicação usado para transmitir informação, tanto no TDM quanto no FDM, pode ser tanto com cabos ou sem cabos (Wireless ou não).

    \subsection{Comutação de pacotes}
    $\rightarrow$ Esse novo conceito permite que mais usuários se comunicando usando a rede. 
    \\Supondo que casa usuário quando ativo transmite a 100kbps e fica ativo à 10\% do tempo total, se houvessem 35 usuários nessa mesma rede com a mesma velocidade, a probabilidade de mais que 10 usuários estejam ativos ao mesmo tempo é menor do que 0,004s.

    \subsection{CP vs CI}
    \textbf{Pontos fortes da CP:}
    \\$\rightarrow$ É excelente para transmissão de rajada (ou seja, envia uma grande quantidade de dados e depois fica em silêncio).
    \\$\rightarrow$ Uso compartilhado de recursos.
    \\$\rightarrow$ É mais simples, não precisa estabelecer uma chamada (reservar os recursos para única e exclusivamente para aqueles dois dispositivos se comunicarem).
    \\~\\\textbf{Pontos fracos da CP:}
    \\$\rightarrow$ Pode haver congestionamento excessivo durante a transmissão (mesmo que seja de probabilidade mínima, ainda pode acontecer). 
    \\Para isso eu preciso de protocoles que garantam a transferência de dados sem erros ou falta de informações, além disso para controlar o congestionamento.
    \\~\\Pergunta: E qual o impacto do congestionamento? 
    \\Resposta: O atraso e perda de pacotes.
    \\~\\\textbf{Pontos fortes da CI:}
    \\$\rightarrow$ Garantia de desempenho (usando toda a banda necessária), já que o canal de comunicação é exclusivo.
\bibliography{references}
\end{document}